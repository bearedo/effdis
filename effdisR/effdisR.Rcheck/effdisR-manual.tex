\nonstopmode{}
\documentclass[letterpaper]{book}
\usepackage[times,inconsolata,hyper]{Rd}
\usepackage{makeidx}
\usepackage[utf8,latin1]{inputenc}
% \usepackage{graphicx} % @USE GRAPHICX@
\makeindex{}
\begin{document}
\chapter*{}
\begin{center}
{\textbf{\huge Package `effdisR'}}
\par\bigskip{\large \today}
\end{center}
\begin{description}
\raggedright{}
\item[Type]\AsIs{Package}
\item[Title]\AsIs{Analysis, visualisation and effort estimate using the fishing
effort databases maintained by ICCAT (International Commission
for the Conservation of Atlantic Tunas)}
\item[Version]\AsIs{1.0}
\item[Date]\AsIs{2015-09-29}
\item[Author]\AsIs{Doug Beare [aut]}
\item[Maintainer]\AsIs{Doug Beare }\email{doug.beare@gmail.com}\AsIs{}
\item[Description]\AsIs{The International Commission for the Conservation of Atlantic Tunas (ICCAT) (www.iccat.int) maintains a database of fishing effort and catches distributed by time-area strata which is known as EFFDIS. A  total of 27 different fishing nations submit catch and effort data to ICCAT for the main gears they use for targeting tuna and tuna-like species within the ICCAT convention area. EFFDIS data are available in two main groups termed, Task 1 and Task 2. Task 1 data are annual totals for catch (eg. tons bluefin tuna caught in 1999 by Japan) by gear in the various relevant 'regions' (Atlantic and Mediterranean) and are believed to be totally comprehensive. Task 2 data, on the other hand, are much more detailed, available at greater spatial (e.g. 5x5 degree square grid) and temporal (e.g. month and year) resolution. The negative side is that they tend to be only partially complete. Comprehensive estimates of fishing effort can, therefore, potentilly be made by 'raising' the Task 2 estimates by those from Task 1.  The EFFDIS database thus represents a rich and valuable source of information on fishing activity in the Atlantic and Mediterranean since 1950. It has the potential to reveal both seasonal and long-term changes in the distributions of the fisheries, and their target species in addition to exposing the vulnerability of various by-catch taxa such as turtles and seabirds. ICCAT contracted Doug Beare (Globefish Consultancy Services, GCS) to develop a modeling approach to estimating overall Atlantic fishing effort exploiting the spatio-temporal information available in the EFFDIS data and the result is the effdisR package}
\item[License]\AsIs{GPL (>=2)}
\end{description}
\Rdcontents{\R{} topics documented:}
\inputencoding{utf8}
\HeaderA{effdisR-package}{Analysis, visualisation and effort estimate using the fishing effort databases maintained by ICCAT (International Commission for the Conservation of Atlantic Tunas)}{effdisR.Rdash.package}
\aliasA{effdisR}{effdisR-package}{effdisR}
\keyword{package}{effdisR-package}
%
\begin{Description}\relax
The International Commission for the Conservation of Atlantic Tunas (ICCAT) (www.iccat.int) maintains a database of fishing effort and catches distributed by time-area strata which is known as EFFDIS. A  total of 27 different fishing nations submit catch and effort data to ICCAT for the main gears they use for targeting tuna and tuna-like species within the ICCAT convention area. EFFDIS data are available in two main groups termed, Task 1 and Task 2. Task 1 data are annual totals for catch (eg. tons bluefin tuna caught in 1999 by Japan) by gear in the various relevant 'regions' (Atlantic and Mediterranean) and are believed to be totally comprehensive. Task 2 data, on the other hand, are much more detailed, available at greater spatial (e.g. 5x5 degree square grid) and temporal (e.g. month and year) resolution. The negative side is that they tend to be only partially complete. Comprehensive estimates of fishing effort can, therefore, potentilly be made by 'raising' the Task 2 estimates by those from Task 1.  The EFFDIS database thus represents a rich and valuable source of information on fishing activity in the Atlantic and Mediterranean since 1950. It has the potential to reveal both seasonal and long-term changes in the distributions of the fisheries, and their target species in addition to exposing the vulnerability of various by-catch taxa such as turtles and seabirds. ICCAT contracted Doug Beare (Globefish Consultancy Services, GCS) to develop a modeling approach to estimating overall Atlantic fishing effort exploiting the spatio-temporal information available in the EFFDIS data and the result is the effdisR package
\end{Description}
%
\begin{Details}\relax
The DESCRIPTION file:
This package was not yet installed at build time.\\{}

Index:  This package was not yet installed at build time.\\{}
An overview of how to use the package, including the most important functions
\end{Details}
%
\begin{Author}\relax
Doug Beare [aut]

Maintainer: Doug Beare <doug.beare@gmail.com>
\end{Author}
%
\begin{References}\relax
\textasciitilde{}\textasciitilde{} Literature or other references for background information \textasciitilde{}\textasciitilde{}
\end{References}
%
\begin{SeeAlso}\relax
\textasciitilde{}\textasciitilde{} Optional links to other man pages, e.g. \textasciitilde{}\textasciitilde{}
\textasciitilde{}\textasciitilde{} \code{\LinkA{<pkg>}{<pkg>}} \textasciitilde{}\textasciitilde{}
\end{SeeAlso}
%
\begin{Examples}
\begin{ExampleCode}
 simple examples of the most important functions 
\end{ExampleCode}
\end{Examples}
\inputencoding{utf8}
\HeaderA{add.covariates}{Function to add depth, sst, chlorophyll a and primary production to a data frame }{add.covariates}
\keyword{\textbackslash{}textasciitilde{}kwd1}{add.covariates}
\keyword{\textbackslash{}textasciitilde{}kwd2}{add.covariates}
%
\begin{Description}\relax

This was set up to investigate whether it was worth using coveriates (other than latitude, longitude, trend, and month) such as bottom depth, sea-surface temperature etc. These covariates are climatological means from NOAA's websites.


\end{Description}
%
\begin{Usage}
\begin{verbatim}
add.covariates(input = t2ce_lf_ll, what.dsn = "effdis-tuna-cc1")
\end{verbatim}
\end{Usage}
%
\begin{Arguments}
\begin{ldescription}
\item[\code{input}] 
Task II data frame in the long format.

\item[\code{what.dsn}] 
DSN for database.

\end{ldescription}
\end{Arguments}
%
\begin{Details}\relax
In linux the DSN is set up in the file etc/odbc.ini, in Windows you need to install the PostgreSQL drivers first and set up the DSN from the Control Panel -> Administrative Tools. See functions get.effdis.t2.data and get.effdis.t1.data for more details.

\end{Details}
%
\begin{Value}
Output is a new data frame with columns added for depth, sea surface temperature, primary production and chlorophyll a concentration.





\end{Value}
%
\begin{Author}\relax
Doug Beare
\end{Author}
%
\begin{Examples}
\begin{ExampleCode}
require(RODBC)
#lln  <- get.effdis.t2.data(which.dsn='effdis-tuna-cc1',which.gear='LL',which.flag='Brasil',which.dsettype = 'n-') # Get data from cloud server
#require(reshape2)
#llnlf <- convert.to.long.format.t2(input = lln) # Convert to long format
#llnlf <- add.covariates(input=llnlf) # Add on covariates



## The function is currently defined as
function (input = t2ce_lf_ll, what.dsn = "effdis-tuna-cc1") 
{
    chan <- odbcConnect(what.dsn, case = "postgresql", believeNRows = FALSE)
    t2ce_distinct_locations_covariates <- sqlQuery(chan, "SELECT *, ST_AsText(the_geom_4326) AS the_point from t2ce_distinct_locations_covariates;")
    tdlc <- SpatialPointsDataFrame(cbind(x = an(ac(t2ce_distinct_locations_covariates$longitude)), 
        y = an(ac(t2ce_distinct_locations_covariates$lat))), 
        data = t2ce_distinct_locations_covariates)
    geogWGS84 <- CRS("+proj=longlat +ellps=WGS84 +datum=WGS84 +no_defs")
    tdlc@proj4string <- geogWGS84
    input$depth_m <- tdlc@data$depth_m[match(paste(input$longitude, 
        input$latitude), paste(tdlc@data$longitude, tdlc@data$latitude))]
    input$m_ann_sst <- tdlc@data$m_ann_sst[match(paste(input$longitude, 
        input$latitude), paste(tdlc@data$longitude, tdlc@data$latitude))]
    input$aq_prim_pro <- tdlc@data$aq_prim_pro[match(paste(input$longitude, 
        input$latitude), paste(tdlc@data$longitude, tdlc@data$latitude))]
    input$m_ann_chla <- tdlc@data$m_ann_chla[match(paste(input$longitude, 
        input$latitude), paste(tdlc@data$longitude, tdlc@data$latitude))]
    out <- input
    out
  }
\end{ExampleCode}
\end{Examples}
\inputencoding{utf8}
\HeaderA{aggt2data}{Function to aggregate Task II data as an alternative to modeling}{aggt2data}
\keyword{\textbackslash{}textasciitilde{}kwd1}{aggt2data}
\keyword{\textbackslash{}textasciitilde{}kwd2}{aggt2data}
%
\begin{Description}\relax
The effdisR library allows user to model (statistically) the catch and effort data as functions of location and time. This function, however, allows the user to simply calculate the effort metrics directly from the raw data. To that end this function simply sums the Task II data over variables of location and time and the result can then be merged with Task I in exacly the same way as the  model output works
\end{Description}
%
\begin{Usage}
\begin{verbatim}
aggt2data(input = lllf, what.flag = "Japan", what.effort = "NO.HOOKS")
\end{verbatim}
\end{Usage}
%
\begin{Arguments}
\begin{ldescription}
\item[\code{input}] 
Task II data in the long format

\item[\code{what.flag}] 
Select flag country, e.g. Japan

\item[\code{what.effort}] 
Select type of effort, e.g. NO.HOOKS

\end{ldescription}
\end{Arguments}
%
\begin{Author}\relax
Doug Beare
\end{Author}
%
\begin{Examples}
\begin{ExampleCode}

#totals <- agg2data(input=lllf,what.flag="Chinese Taipei", what.effort = "NO.HOOKS") # Total catches by location, year, month and species for selected flag country


## The function is currently defined as
function (input = lllf, what.flag = "Japan", what.effort = "NO.HOOKS") 
{
    input_kgs <- input[input$catchunit == "kg" & input$eff1type == 
        what.effort & lllf$flagname == what.flag, ]
    allt2 <- aggregate(list(eff = input_kgs$eff1, measured_catch = input_kgs$measured_catch), 
        by = list(year = input_kgs$year, month = input_kgs$month, 
            longitude = input_kgs$longitude, latitude = input_kgs$latitude, 
            species = input_kgs$species, flagname = input_kgs$flagname), 
        sum, na.rm = T)
    allt2
  }
\end{ExampleCode}
\end{Examples}
\inputencoding{utf8}
\HeaderA{convert2long.format.t2}{Function to convert short format to long format which is useful for regression modeling}{convert2long.format.t2}
\keyword{\textbackslash{}textasciitilde{}kwd1}{convert2long.format.t2}
\keyword{\textbackslash{}textasciitilde{}kwd2}{convert2long.format.t2}
%
\begin{Description}\relax
Takes the task II data in the form t2ce, selects the most useful columns and converts them to the long-format. Instead of nine columns for each species there is now a single column for 'species'
\end{Description}
%
\begin{Usage}
\begin{verbatim}
convert2long.format.t2(input = t2ce)
\end{verbatim}
\end{Usage}
%
\begin{Arguments}
\begin{ldescription}
\item[\code{input}] 
Data frame of task II data in the short format, ie. t2ce from the PostgreSQL database

\end{ldescription}
\end{Arguments}
%
\begin{Details}\relax
This is based on melt from reshape 2 and is simply handy for regression modeling and often it's easier to write SQL queries with data in the long-format
\end{Details}
%
\begin{Value}
Returns a new data frame in the long format with column names as per input but all the species columns are named,'species' and the catches are named, 'measured\_catch'





\end{Value}
%
\begin{Author}\relax
Doug Beare
\end{Author}
%
\begin{Examples}
\begin{ExampleCode}


#lllf <- convert2long.format.t2(input =ll1) # Convert longline data in the short format to long format.

## The function is currently defined as
function (input = t2ce) 
{
    tdata <- input
    task2.simple <- data.frame(year = tdata$year, trend = tdata$trend, 
        month = tdata$month, region = tdata$region, flagname = tdata$flagname, 
        fleetcode = tdata$fleetcode, geargrpcode = tdata$geargrpcode, 
        longitude = tdata$longitude, latitude = tdata$latitude, 
        eff1 = tdata$eff1, eff1type = tdata$eff1type, dsettype = tdata$dsettype, 
        catchunit = tdata$catchunit, alb = tdata$alb, bft = tdata$bft, 
        bet = tdata$bet, skj = tdata$skj, yft = tdata$yft, swo = tdata$swo, 
        bum = tdata$bum, sai = tdata$sai, whm = tdata$whm, tot9sp = tdata$totsp9)
    task2.lf <- melt(task2.simple[, -23], id = c("year", "trend", 
        "month", "region", "flagname", "fleetcode", "geargrpcode", 
        "longitude", "latitude", "eff1", "eff1type", "dsettype", 
        "catchunit"))
    dimnames(task2.lf)[[2]][14:15] <- c("species", "measured_catch")
    task2.lf
  }
\end{ExampleCode}
\end{Examples}
\inputencoding{utf8}
\HeaderA{find.ocean}{Function to determine which Ocean an observation is in}{find.ocean}
\keyword{\textbackslash{}textasciitilde{}kwd1}{find.ocean}
\keyword{\textbackslash{}textasciitilde{}kwd2}{find.ocean}
%
\begin{Description}\relax
Adds an index/flag to the input data frame denoting whether an observation is in the Pacific, Atlantic, Mediterranean or on land
\end{Description}
%
\begin{Usage}
\begin{verbatim}
find.ocean(input = grd)
\end{verbatim}
\end{Usage}
%
\begin{Arguments}
\begin{ldescription}
\item[\code{input}] 
A data frame containing a column of longitudes and latitudes.

\end{ldescription}
\end{Arguments}
%
\begin{Details}\relax

The shape file dataset (World\_Seas) used by this function represents the boundaries of the major oceans and seas of the world. The source for the boundaries is the publication 'Limits of Oceans \& Seas, Special Publication No. 23' published by the IHO in 1953. The dataset was composed by the Flanders Marine Data and Information Centre and are available for download here, http://www.marineregions.org/ . The definitions of 'Atlantic' used by ICCAT are rather crude. Thus function tries to mimic ICCAT definitions and 'Atlantic' for example includes the Gulf of St Lawrence, the Caribbean Sea, The Celtic Sea, The Gulf of Guinea, The Gulf of Mexico and The Irish Sea. Obviously this is flexible and can easily be changed



\end{Details}
%
\begin{Value}
Data frame with column called 'which.ocean' which contains strings: 'atl', 'pac', 'med' and 'land'
\end{Value}
%
\begin{Author}\relax
Doug Beare
\end{Author}
%
\begin{References}\relax
http://www.marineregions.org/
\end{References}
%
\begin{Examples}
\begin{ExampleCode}

#llnw <- get.effdis.t2.data(which.dsn='effdis-tuna-cc1',which.gear='LL',which.flag='All',which.dsettype = 'nw') # Get data from PostgreSQL
#llnw <- find.ocean(input=llnw) # Add which.ocean variable


## The function is currently defined as
function (input = grd) 
{
    seas <- readOGR(dsn = "/home/doug/effdis/data", layer = "World_Seas")
    geogWGS84 <- CRS("+proj=longlat +ellps=WGS84 +datum=WGS84 +no_defs")
    seas@proj4string <- geogWGS84
    seas.polys <- as.character(sort(unique(seas@data$NAME)))
    wo <- grep("Atl", seas.polys)
    wi <- grep("Med", seas.polys)
    wj <- grep("Adriatic", seas.polys)
    wk <- grep("Aegean", seas.polys)
    wl <- grep("Balearic", seas.polys)
    wm <- grep("Bay of Biscay", seas.polys)
    wn <- grep("Bristol", seas.polys)
    wp <- grep("Caribbean", seas.polys)
    wq <- grep("Celtic", seas.polys)
    wr <- grep("English Channel", seas.polys)
    ws <- grep("Lawrence", seas.polys)
    wt <- grep("Inner Seas", seas.polys)
    wu <- grep("Ionian", seas.polys)
    wv <- grep("Irish", seas.polys)
    wx <- grep("North Sea", seas.polys)
    wz <- grep("Gibra", seas.polys)
    wa <- grep("Ligurian", seas.polys)
    wzz <- grep("Tyrr", seas.polys)
    wxx <- grep("Alb", seas.polys)
    wmm <- grep("Mex", seas.polys)
    wpa <- grep("Pacific", seas.polys)
    gog <- grep("Guin", seas.polys)
    atlantic <- seas[seas@data$NAME %in% seas.polys[c(wo, wp, 
        ws, wv, wmm, gog)], ]
    med <- seas[seas@data$NAME %in% seas.polys[c(wi, wj, wk, 
        wl, wu, wz, wa, wzz, wxx)], ]
    pacific <- seas[seas@data$NAME %in% seas.polys[c(wpa)], ]
    input.spdf <- SpatialPointsDataFrame(cbind(x = an(ac(input$longitude)), 
        y = an(ac(input$latitude))), data = input)
    geogWGS84 <- CRS("+proj=longlat +ellps=WGS84 +datum=WGS84 +no_defs")
    input.spdf@proj4string <- geogWGS84
    idx.atl <- over(input.spdf, atlantic)
    idx.med <- over(input.spdf, med)
    idx.pac <- over(input.spdf, pacific)
    which.ocean <- rep(NA, length(input[, 1]))
    which.ocean[which(!is.na(idx.atl[, 1]))] <- "atl"
    which.ocean[which(!is.na(idx.med[, 1]))] <- "med"
    which.ocean[which(!is.na(idx.pac[, 1]))] <- "pac"
    which.ocean[is.na(which.ocean)] <- "land"
    input$which.ocean <- which.ocean
    input
  }
\end{ExampleCode}
\end{Examples}
\inputencoding{utf8}
\HeaderA{fit2stageGAMtoCatch}{Function to fit generalised additive models to the EFFDIS catch data in 2 stages. The first is a Bernoulli model the second is a Gamma model fitted to the positive part of the data}{fit2stageGAMtoCatch}
\keyword{\textbackslash{}textasciitilde{}kwd1}{fit2stageGAMtoCatch}
\keyword{\textbackslash{}textasciitilde{}kwd2}{fit2stageGAMtoCatch}
%
\begin{Description}\relax
The function extracts a relevant subset of the Task II data. For the first model a binary variable is constructed where zero catch = 0 and positive catch = 1 which is modeled as a function of longitude, latitude, trend and season using a GAM (mgcv library) from the quasibinomial family.  In the second model the catches that = 0 are removed and the positive component is modeled using a GAM (mgcv library) from the Gamma family. In the call se

The seasonal component is estimated using harmonics.
\end{Description}
%
\begin{Usage}
\begin{verbatim}
fit2stageGAMtoCatch(input = pslf, which.species = "bft", start.year = 1950, end.year = 2015, which.flag = "Japan", kk = 6)
\end{verbatim}
\end{Usage}
%
\begin{Arguments}
\begin{ldescription}
\item[\code{input}] 
Task II data in the long format

\item[\code{which.species}] 
Select which species to model, e.g. alb or skj

\item[\code{start.year}] 
Select from which year you want to start

\item[\code{end.year}] 
Select from which year you want to end

\item[\code{which.flag}] 
Select which flag state, e.g. Japan

\item[\code{kk}] 
Select smoothing parameter. Default is 6. 

\end{ldescription}
\end{Arguments}
%
\begin{Value}

\begin{ldescription}
\item[\code{pmod}] Model object from the binary GAM
\item[\code{pmod.data }] Data used in the binary GAM fit
\item[\code{gmod}] Model object from the Gamma GAM fitted to positive data
\item[\code{gmod.data}] Data used in the Gamma GAM fitted to positive data

\end{ldescription}
\end{Value}
%
\begin{Author}\relax
Doug Beare
\end{Author}
%
\begin{References}\relax
https://cran.r-project.org/web/packages/mgcv/index.html
\end{References}
%
\begin{Examples}
\begin{ExampleCode}

#alb <- fit2stageGAMtoCatch(input=lllf,which.flag='Japan',which.species='alb',start.year=1995,end.year=2010)


## The function is currently defined as
function (input = pslf, which.species = "bft", start.year = 1950, 
    end.year = 2015, which.flag = "Japan", kk = 6) 
{
    input <- input[input$species == which.species, ]
    input <- input[input$year >= start.year & input$year <= end.year, 
        ]
    if (which.flag == "All") {
        input <- input
        print("Modeling all data")
    }
    else {
        input <- input[input$flagname == which.flag, ]
    }
    bin <- ifelse(input$measured_catch == 0, 0, 1)
    input$bin <- bin
    tbin <- table(input$bin)
    if (sum(bin) >= 10) {
        input <- input[input$catchunit == "kg", ]
        input$lmeasured_catch <- log(input$measured_catch + 1)
        bbs <- "cr"
        dat0 <- input
        ss = cc = matrix(NA, nr = length(dat0[, 1]), nc = 6)
        for (i in 1:6) {
            cc[, i] <- cos(2 * pi * i * dat0$trend/12)
            ss[, i] <- sin(2 * pi * i * dat0$trend/12)
        }
        ss <- ss[, -6]
        dat1 <- cbind(dat0, ss, cc)
        dd <- dim(dat0)
        dimnames(dat1)[[2]][(dd[2] + 1):(dim(dat1)[2])] <- c(paste("sin", 
            1:5, sep = ""), paste("cos", 1:6, sep = ""))
        input <- dat1
        b1 <- gam(bin ~ te(longitude, latitude, k = 6, bs = bbs) + 
            te(trend, k = 6, bs = bbs) + sin1 + cos1 + sin2 + 
            cos2 + sin3 + cos3 + sin4 + cos4 + sin5 + cos5 + 
            cos6, data = input, family = quasibinomial(link = "logit"), 
            method = "REML", select = TRUE)
        input1 <- aggregate(list(measured_catch = input$measured_catch), 
            by = list(trend = input$trend, month = input$month, 
                longitude = input$longitude, latitude = input$latitude), 
            sum, na.rm = T)
        bin <- ifelse(input1$measured_catch == 0, 0, 1)
        input1$bin <- bin
        input2 <- input1[input1$bin == 1, ]
        dat0 <- input2
        ss = cc = matrix(NA, nr = length(dat0[, 1]), nc = 6)
        for (i in 1:6) {
            cc[, i] <- cos(2 * pi * i * dat0$trend/12)
            ss[, i] <- sin(2 * pi * i * dat0$trend/12)
        }
        ss <- ss[, -6]
        dat1 <- cbind(dat0, ss, cc)
        dd <- dim(dat0)
        dimnames(dat1)[[2]][(dd[2] + 1):(dim(dat1)[2])] <- c(paste("sin", 
            1:5, sep = ""), paste("cos", 1:6, sep = ""))
        input2 <- dat1
        g1 <- gam(measured_catch ~ te(longitude, latitude, k = kk, 
            bs = bbs) + te(trend, k = kk, bs = bbs) + sin1 + 
            cos1 + sin2 + cos2 + sin3 + cos3 + sin4 + cos4 + 
            sin5 + cos5 + cos6, family = Gamma(link = "log"), 
            method = "REML", select = TRUE, data = input2)
        gc(reset = T)
        mods <- list(pmod = b1, pmod.data = input, gmod = g1, 
            gmod.data = input2)
        mods
    }
    else {
        print("Insufficient data to support model")
    }
  }
\end{ExampleCode}
\end{Examples}
\inputencoding{utf8}
\HeaderA{fitGAMtoEffort}{Function to fit generalised additive model from the quasipoisson family to the effort data}{fitGAMtoEffort}
\keyword{\textbackslash{}textasciitilde{}kwd1}{fitGAMtoEffort}
\keyword{\textbackslash{}textasciitilde{}kwd2}{fitGAMtoEffort}
%
\begin{Description}\relax
The function extracts a relevant subset of the Task II data and models the effort data selected (e.g. NO.HOOKS) as a function of location (longitude and latitude) and time (trend and season). Seasonality is modeled using harmonics. 
\end{Description}
%
\begin{Usage}
\begin{verbatim}
fitGAMtoEffort(input = lllf, which.flag = "Japan", which.effort = "NO.HOOKS", start.year = 1950, end.year = 2010, kk = 6)
\end{verbatim}
\end{Usage}
%
\begin{Arguments}
\begin{ldescription}
\item[\code{input}] 
Task II data in the long format

\item[\code{which.flag}] 
Select which flag state, e.g. Japan

\item[\code{which.effort}] 
Select effort type, e.g. NO.HOOKS

\item[\code{start.year}] 
Select from which year you want to start

\item[\code{end.year}] 
Select from which year you want to end

\item[\code{kk}] 
Select smoothing parameter. Default is 6. 

\end{ldescription}
\end{Arguments}
%
\begin{Details}\relax
The harmonic regression is set up according to methods described in Introductory Time Series with R, 
http://www.springer.com/us/book/9780387886978

\end{Details}
%
\begin{Value}

\begin{ldescription}
\item[\code{emod}] Model object from the quasipoisson GAM
\item[\code{pmod.data }] Data used in the quasipoisson GAM
\end{ldescription}
\end{Value}
%
\begin{Author}\relax
Doug Beare
\end{Author}
%
\begin{References}\relax
https://cran.r-project.org/web/packages/mgcv/index.html
\end{References}
%
\begin{Examples}
\begin{ExampleCode}

emod <- fitGAMtoEffort(input=lllf,which.flag='Japan',which.effort='NO.HOOKS',start.year=1970,end.year=2010,which.gam='gam')



## The function is currently defined as
function (input = lllf, which.flag = "Japan", which.effort = "NO.HOOKS", 
    start.year = 1950, end.year = 2010, kk = 6) 
{
    if (which.flag == "All") {
        input <- input
        print("Modeling all data")
    }
    else {
        input <- input[input$flagname == which.flag, ]
    }
    n0 <- input[input$dsettype == "n-", ]
    nw <- input[input$dsettype == "nw", ]
    mm <- duplicated(nw[, c(1:11)])
    nw <- nw[mm == TRUE, ]
    w0 <- input[input$dsettype == "-w", ]
    input1 <- rbind(n0, nw, w0)
    input2 <- aggregate(list(eff1 = input1$eff1), by = list(trend = input1$trend, 
        month = input1$month, longitude = input1$longitude, latitude = input1$latitude), 
        sum)
    dat0 <- input2
    ss = cc = matrix(NA, nr = length(dat0[, 1]), nc = 6)
    for (i in 1:6) {
        cc[, i] <- cos(2 * pi * i * dat0$trend/12)
        ss[, i] <- sin(2 * pi * i * dat0$trend/12)
    }
    ss <- ss[, -6]
    dat1 <- cbind(dat0, ss, cc)
    dd <- dim(dat0)
    dimnames(dat1)[[2]][(dd[2] + 1):(dim(dat1)[2])] <- c(paste("sin", 
        1:5, sep = ""), paste("cos", 1:6, sep = ""))
    input2 <- dat1
    bbs <- "cr"
    h1 <- gam(eff1 ~ te(longitude, latitude, k = kk, bs = bbs) + 
        te(trend, k = kk, bs = bbs) + sin1 + cos1 + sin2 + cos2 + 
        sin3 + cos3 + sin4 + cos4 + sin5 + cos5 + cos6, family = quasipoisson(link = "log"), 
        method = "REML", data = input2)
    out <- list(emod = h1, emod.data = input2)
    out
  }
\end{ExampleCode}
\end{Examples}
\inputencoding{utf8}
\HeaderA{get.effdis.t1.data}{Function to get the Task I data from the PostgreSQL database on the ICCAT cloud server}{get.effdis.t1.data}
\keyword{\textbackslash{}textasciitilde{}kwd1}{get.effdis.t1.data}
\keyword{\textbackslash{}textasciitilde{}kwd2}{get.effdis.t1.data}
%
\begin{Description}\relax
Extracts Task I data from the PostgreSQL database (effdis) on the ICCAT cloud server
\end{Description}
%
\begin{Usage}
\begin{verbatim}
get.effdis.t1.data(which.dsn = "effdis-tuna-cc1", which.gear = "LL", which.region = "AT", which.flag = "All", which.datatype = "C")
\end{verbatim}
\end{Usage}
%
\begin{Arguments}
\begin{ldescription}
\item[\code{which.dsn}] 
Data source name, see details.

\item[\code{which.gear}] 
Select gear type, e.g. PS=Purse seine

\item[\code{which.region}] 
Select region, e.g. AT = Atantic

\item[\code{which.flag}] 
Select flag naton, e.g. Japan

\item[\code{which.datatype}] 
Select datatype. C=Catch, D=Discard.

\end{ldescription}
\end{Arguments}
%
\begin{Details}\relax
which.dsn is a text string telling R which datbase to connect to. You need to have the PostgreSQL drivers installed and the simplest way to do this is install PostgreSQL. In linux you will then need to edit the 
etc/odbc.ini file, see example odbc.ini below. Hence which.dsn = 'effdis-tuna-cc1'

[ODBC Data Sources]
effdis = effdis-tuna-cc1

[effdis-tuna-cc1]
Driver = /usr/lib/x86\_64-linux-gnu/odbc/psqlodbcw.so
Database = effdis
Servername = 134.213.29.249
Username = postgres
Password = Postgres1
Protocol = 8.2.5
ReadOnly = 0

[ODBC]
InstallDir = /usr/lib

\end{Details}
%
\begin{Value}
Data frame of the Task I data extracted from the database.
\end{Value}
%
\begin{Author}\relax
Doug Beare
\end{Author}
%
\begin{References}\relax
http://www.r-bloggers.com/getting-started-with-postgresql-in-r/
\end{References}
%
\begin{Examples}
\begin{ExampleCode}

#ll.t1 <- get.effdis.t1.data.r(which.dsn='effdis-tuna-cc1',which.gear = 'LL',which.region='AT',which.flag='Japan')


\end{ExampleCode}
\end{Examples}
\inputencoding{utf8}
\HeaderA{get.effdis.t2.data}{Function to get the Task II data from the postgreSQL database on the ICCAT cloud server}{get.effdis.t2.data}
\keyword{\textbackslash{}textasciitilde{}kwd1}{get.effdis.t2.data}
\keyword{\textbackslash{}textasciitilde{}kwd2}{get.effdis.t2.data}
%
\begin{Description}\relax
Extracts Task I data from the PostgreSQL database (effdis) on the ICCAT cloud server
\end{Description}
%
\begin{Usage}
\begin{verbatim}
get.effdis.t2.data(which.dsn = "effdis-tuna-cc1", which.gear = "LL", which.region = "AT", which.flag = "All", which.dsettype = "-w")
\end{verbatim}
\end{Usage}
%
\begin{Arguments}
\begin{ldescription}
\item[\code{which.dsn}] 
Data source name, see details.

\item[\code{which.gear}] 
Select gear type, e.g. PS=Purse seine

\item[\code{which.region}] 
Select region, e.g. AT = Atantic

\item[\code{which.flag}] 
Select flag naton, e.g. Japan

\item[\code{which.dsettype}] 
Select dsettype. 'n-' = numbers, '-w' = weight in kgs, 'nw'= both numbers and weights collected.

\end{ldescription}
\end{Arguments}
%
\begin{Details}\relax
which.dsn is a text string string telling R which datbase to connect to. You need to have the PostgreSQL drivers installed and the simplest way to do this is install PostgreSQL. In linux you will then need to edit the 
etc/odbc.ini file, see example odbc.ini below. Hence which.dsn = 'effdis-tuna-cc1'

[ODBC Data Sources]
effdis = effdis-tuna-cc1

[effdis-tuna-cc1]
Driver = /usr/lib/x86\_64-linux-gnu/odbc/psqlodbcw.so
Database = effdis
Servername = 134.213.29.249
Username = postgres
Password = Postgres1
Protocol = 8.2.5
ReadOnly = 0

[ODBC]
InstallDir = /usr/lib

\end{Details}
%
\begin{Value}
Data frame of the Task II data extracted from the database.
\end{Value}
%
\begin{Author}\relax
Doug Beare
\end{Author}
%
\begin{References}\relax
http://www.r-bloggers.com/getting-started-with-postgresql-in-r/
\end{References}
%
\begin{Examples}
\begin{ExampleCode}
#llnw <- get.effdis.t2.data(which.dsn='effdis-tuna-cc1',which.gear='LL',which.flag='All',which.dsettype = 'nw')
\end{ExampleCode}
\end{Examples}
\inputencoding{utf8}
\HeaderA{kgs.from.nos}{Function to estimate weight caught for countries that only record numbers}{kgs.from.nos}
\keyword{\textbackslash{}textasciitilde{}kwd1}{kgs.from.nos}
\keyword{\textbackslash{}textasciitilde{}kwd2}{kgs.from.nos}
%
\begin{Usage}
\begin{verbatim}
kgs.from.nos(input = pslf)
\end{verbatim}
\end{Usage}
%
\begin{Arguments}
\begin{ldescription}
\item[\code{input}] 


\end{ldescription}
\end{Arguments}
%
\begin{Examples}
\begin{ExampleCode}
##---- Should be DIRECTLY executable !! ----
##-- ==>  Define data, use random,
##--	or do  help(data=index)  for the standard data sets.

## The function is currently defined as
function (input = pslf) 
{
    t2ce_lf_kg <- input[input$dsettype %in% c("-w", "nw"), ]
    t2ce_lf_kg <- t2ce_lf_kg[t2ce_lf_kg$catchunit != "--", ]
    t2ce_lf_nr <- input[input$dsettype == "n-", ]
    t2ce_lf_nr$measured_catch_nr <- t2ce_lf_nr$measured_catch
    dd <- dim(t2ce_lf_nr)
    if (dd[1] > 0) {
        t2ce_lf_nr$lnr <- log(t2ce_lf_nr$measured_catch_nr)
        aa <- exp(predict(bm, t2ce_lf_nr, type = "response"))
        ww <- (1:length(t2ce_lf_nr[, 1]))[t2ce_lf_nr$measured_catch == 
            0]
        aa[ww] <- 0
        t2ce_lf_nr$measured_catch <- aa
        t2ce_lf_nr$dsettype <- "-w"
        t2ce_lf_nr$catchunit <- "kg"
        t2ce_lf_nr <- t2ce_lf_nr[, -c(16, 17)]
        t2ce_lf <- rbind(t2ce_lf_kg, t2ce_lf_nr)
        t2ce_lf
    }
    else {
        print("All data are in kgs")
    }
  }
\end{ExampleCode}
\end{Examples}
\inputencoding{utf8}
\HeaderA{model.nos.kgs}{Function to fit linear model (kgs as a function of numbers) for those flags that send both}{model.nos.kgs}
\keyword{\textbackslash{}textasciitilde{}kwd1}{model.nos.kgs}
\keyword{\textbackslash{}textasciitilde{}kwd2}{model.nos.kgs}
%
\begin{Usage}
\begin{verbatim}
model.nos.kgs(input = pslf, which.dsn = "effdis-local", which.gear = "LL")
\end{verbatim}
\end{Usage}
%
\begin{Arguments}
\begin{ldescription}
\item[\code{input}] 


\item[\code{which.dsn}] 


\item[\code{which.gear}] 


\end{ldescription}
\end{Arguments}
%
\begin{Examples}
\begin{ExampleCode}
##---- Should be DIRECTLY executable !! ----
##-- ==>  Define data, use random,
##--	or do  help(data=index)  for the standard data sets.

## The function is currently defined as
function (input = pslf, which.dsn = "effdis-local", which.gear = "LL") 
{
    pp0 <- aggregate(list(measured_catch = input$measured_catch), 
        by = list(trend = input$trend, month = input$month, flagname = input$flagname, 
            catchunit = input$catchunit, species = input$species), 
        sum, na.rm = T)
    pp0_nr <- pp0[pp0$catchunit == "nr", ]
    dimnames(pp0_nr)[[2]][6] <- "measured_catch_nr"
    pp0_kg <- pp0[pp0$catchunit == "kg", ]
    dimnames(pp0_kg)[[2]][6] <- "measured_catch_kg"
    pp0_kg_nr <- merge(pp0_kg[, -4], pp0_nr[, -4])
    library(lattice)
    xyplot(log(measured_catch_kg) ~ log(measured_catch_nr), groups = flagname, 
        data = pp0_kg_nr[pp0_kg_nr$species == "bft", ])
    xyplot(log(measured_catch_kg) ~ log(measured_catch_nr) | 
        species, auto.key = TRUE, groups = flagname, data = pp0_kg_nr)
    pp0_kg_nr$lnr <- log(pp0_kg_nr$measured_catch_nr)
    pp0_kg_nr$lkg <- log(pp0_kg_nr$measured_catch_kg)
    pp0_kg_nr$lnr[pp0_kg_nr$lnr == "-Inf"] <- NA
    pp0_kg_nr$lkg[pp0_kg_nr$lkg == "-Inf"] <- NA
    m1 <- lm(lkg ~ lnr, data = pp0_kg_nr, na.action = "na.omit")
    m2 <- lm(lkg ~ lnr + trend, data = pp0_kg_nr, na.action = "na.omit")
    m3 <- lm(lkg ~ lnr + trend + species, data = pp0_kg_nr, na.action = "na.omit")
    best.model <- step(m3, direction = "both")
    print(summary(best.model))
    best.model
  }
\end{ExampleCode}
\end{Examples}
\inputencoding{utf8}
\HeaderA{plot.mods}{Function to plot output of fitted GAMs}{plot.mods}
\keyword{\textbackslash{}textasciitilde{}kwd1}{plot.mods}
\keyword{\textbackslash{}textasciitilde{}kwd2}{plot.mods}
%
\begin{Usage}
\begin{verbatim}
plot.mods(input = bft.aa, cmod = bft, which.year = 1995, which.month = 1, grid.res = 5, which.value = "prob", which.gear = "PS", plot.samples.only = TRUE)
\end{verbatim}
\end{Usage}
%
\begin{Arguments}
\begin{ldescription}
\item[\code{input}] 


\item[\code{cmod}] 


\item[\code{which.year}] 


\item[\code{which.month}] 


\item[\code{grid.res}] 


\item[\code{which.value}] 


\item[\code{which.gear}] 


\item[\code{plot.samples.only}] 


\end{ldescription}
\end{Arguments}
%
\begin{Examples}
\begin{ExampleCode}
##---- Should be DIRECTLY executable !! ----
##-- ==>  Define data, use random,
##--	or do  help(data=index)  for the standard data sets.

## The function is currently defined as
function (input = bft.aa, cmod = bft, which.year = 1995, which.month = 1, 
    grid.res = 5, which.value = "prob", which.gear = "PS", plot.samples.only = TRUE) 
{
    min.lat <- min(cmod$pmod.data$latitude)
    max.lat <- max(cmod$pmod.data$latitude)
    min.lon <- min(cmod$pmod.data$longitude)
    max.lon <- max(cmod$pmod.data$longitude)
    which.species <- as.character(input$species[1])
    lonnie <- seq(min.lon, max.lon, by = grid.res)
    lattie <- seq(min.lat, max.lat, by = grid.res)
    lo <- length(lonnie)
    la <- length(lattie)
    if (which.value == "prob") {
        if (plot.samples.only == TRUE) {
            input[, which.value][input$observation == F] <- NA
        }
        image(lonnie, lattie, matrix(input[, which.value][input$year == 
            which.year & input$month == which.month], lo, la), 
            col = topo.colors(100))
        contour(lonnie, lattie, matrix(input[, which.value][input$year == 
            which.year & input$month == which.month], lo, la), 
            add = T)
        map("worldHires", add = T, fill = T)
        title(paste(toupper(which.species), "-", which.value, month.abb[which.month], 
            which.year, which.gear))
        w0 <- (1:length(cmod$pmod.data[, 1]))[cmod$pmod.data$year == 
            which.year & cmod$pmod.data$month == which.month]
        points(cmod$pmod.data$longitude[w0], cmod$pmod.data$latitude[w0], 
            pch = ".")
    }
    else {
        if (plot.samples.only == TRUE) {
            input[, which.value][input$observation == F] <- NA
        }
        image(lonnie, lattie, matrix(log(input[, which.value])[input$year == 
            which.year & input$month == which.month], lo, la), 
            col = topo.colors(100))
        contour(lonnie, lattie, matrix(log(input[, which.value])[input$year == 
            which.year & input$month == which.month], lo, la), 
            add = T)
        map("worldHires", add = T, fill = T)
        title(paste(toupper(which.species), "-", which.value, month.abb[which.month], 
            which.year, which.gear))
        w0 <- (1:length(cmod$pmod.data$longitude))[cmod$pmod.data$year == 
            which.year & cmod$pmod.data$month == which.month]
        points(cmod$pmod.data$longitude[w0], cmod$pmod.data$latitude[w0], 
            pch = ".")
    }
  }
\end{ExampleCode}
\end{Examples}
\inputencoding{utf8}
\HeaderA{predict.effdis.t2.data}{Function to create comprehensive space-time grid and predict data from the GAMs}{predict.effdis.t2.data}
\keyword{\textbackslash{}textasciitilde{}kwd1}{predict.effdis.t2.data}
\keyword{\textbackslash{}textasciitilde{}kwd2}{predict.effdis.t2.data}
%
\begin{Usage}
\begin{verbatim}
predict.effdis.t2.data(cmod = mods, effmod = emod, grid.res = 5, start.year = 1995, end.year = 2010, which.flag = "All", which.gear = "LL")
\end{verbatim}
\end{Usage}
%
\begin{Arguments}
\begin{ldescription}
\item[\code{cmod}] 


\item[\code{effmod}] 


\item[\code{grid.res}] 


\item[\code{start.year}] 


\item[\code{end.year}] 


\item[\code{which.flag}] 


\item[\code{which.gear}] 


\end{ldescription}
\end{Arguments}
%
\begin{Examples}
\begin{ExampleCode}
##---- Should be DIRECTLY executable !! ----
##-- ==>  Define data, use random,
##--	or do  help(data=index)  for the standard data sets.

## The function is currently defined as
function (cmod = mods, effmod = emod, grid.res = 5, start.year = 1995, 
    end.year = 2010, which.flag = "All", which.gear = "LL") 
{
    pmod.data <- cmod$pmod.data
    min.lat <- min(pmod.data$latitude)
    max.lat <- max(pmod.data$latitude)
    min.lon <- min(pmod.data$longitude)
    max.lon <- max(pmod.data$longitude)
    t1 <- min(pmod.data$trend)
    t2 <- max(pmod.data$trend)
    lonnie <- seq(min.lon, max.lon, by = grid.res)
    lattie <- seq(min.lat, max.lat, by = grid.res)
    lo <- length(lonnie)
    la <- length(lattie)
    grd <- data.frame(expand.grid(longitude = lonnie, latitude = lattie))
    grd <- find.ocean.r(input = grd[, c(1, 2)])
    lyrs <- length(start.year:end.year)
    lloc <- lo * la
    ltrnd <- lyrs * 12
    ngrd <- data.frame(longitude = rep(grd$longitude, ltrnd), 
        latitude = rep(grd$latitude, ltrnd), which.ocean = rep(grd$which.ocean, 
            ltrnd), year = rep(start.year:end.year, rep((lo * 
            la * 12), lyrs)), month = rep(rep(1:12, rep(lo * 
            la, 12)), lyrs))
    ngrd$trend <- trend.r(ngrd$year, ngrd$month, start.year = 1950)
    ngrd$flagname <- which.flag
    ngrd$geargrp <- which.gear
    dat0 <- ngrd
    ss = cc = matrix(NA, nr = length(dat0[, 1]), nc = 6)
    for (i in 1:6) {
        cc[, i] <- cos(2 * pi * i * dat0$trend/12)
        ss[, i] <- sin(2 * pi * i * dat0$trend/12)
    }
    ss <- ss[, -6]
    dat1 <- cbind(dat0, ss, cc)
    dd <- dim(dat0)
    dimnames(dat1)[[2]][(dd[2] + 1):(dim(dat1)[2])] <- c(paste("sin", 
        1:5, sep = ""), paste("cos", 1:6, sep = ""))
    ngrd <- dat1
    prob <- predict(cmod$pmod, ngrd, type = "response")
    measured_catch <- predict(cmod$gmod, ngrd, type = "response")
    eff <- predict(effmod$emod, ngrd, type = "response")
    prob[ngrd$which.ocean %in% c("land", "med", "pac")] <- NA
    measured_catch[ngrd$which.ocean %in% c("land", "med", "pac")] <- NA
    eff[ngrd$which.ocean %in% c("land", "med", "pac")] <- NA
    ngrd$prob <- round(as.vector(prob), 3)
    ngrd$measured_catch <- round(as.vector(measured_catch), 3)
    ngrd$eff <- round(as.vector(eff), 3)
    which.species <- as.character(cmod$pmod.data$species[1])
    print(which.species)
    ngrd$species <- which.species
    ngrd$catch <- round(ngrd$prob * ngrd$measured_catch, 3)
    ngrd$cpue <- round(ngrd$catch/ngrd$eff, 3)
    mm.dat <- paste(pmod.data$longitude, pmod.data$latitude, 
        pmod.data$trend, pmod.data$month)
    mm.grd <- paste(ngrd$longitude, ngrd$latitude, ngrd$trend, 
        ngrd$month)
    mm <- match(mm.grd, mm.dat)
    mm <- ifelse(is.na(mm), F, T)
    ngrd$observation <- mm
    model.data <- ngrd
    filename <- paste("model-data-", which.species, "-", which.flag, 
        "-", which.gear, "-", start.year, "-", end.year, ".csv", 
        sep = "")
    write.table(model.data[model.data$observation == TRUE, ], 
        file = filename, sep = ",", row.names = F, col.names = F)
    model.data
  }
\end{ExampleCode}
\end{Examples}
\inputencoding{utf8}
\HeaderA{prepare.effdis.data}{Clean up the EFFDIS data}{prepare.effdis.data}
\keyword{\textbackslash{}textasciitilde{}kwd1}{prepare.effdis.data}
\keyword{\textbackslash{}textasciitilde{}kwd2}{prepare.effdis.data}
%
\begin{Usage}
\begin{verbatim}
prepare.effdis.data(input = data)
\end{verbatim}
\end{Usage}
%
\begin{Arguments}
\begin{ldescription}
\item[\code{input}] 


\end{ldescription}
\end{Arguments}
%
\begin{Examples}
\begin{ExampleCode}
##---- Should be DIRECTLY executable !! ----
##-- ==>  Define data, use random,
##--	or do  help(data=index)  for the standard data sets.

## The function is currently defined as
function (input = data) 
{
    input <- input[input$eff1type != "--", ]
    input <- input[input$catchunit != "--", ]
    input <- input[input$eff1type != "--", ]
    input$catchunit <- as.character(input$catchunit)
    input$dsettype <- as.character(input$dsettype)
    input$flagname <- as.character(input$flagname)
    input$fleetcode <- as.character(input$fleetcode)
    input$geargrpcode <- as.character(input$geargrpcode)
    input$region <- as.character(input$region)
    input$eff1type <- as.character(input$eff1type)
    input
  }
\end{ExampleCode}
\end{Examples}
\inputencoding{utf8}
\HeaderA{spatial.coverage.by.year.task2}{ Function to plot the spatial distribution of EFFDIS data each year}{spatial.coverage.by.year.task2}
\keyword{\textbackslash{}textasciitilde{}kwd1}{spatial.coverage.by.year.task2}
\keyword{\textbackslash{}textasciitilde{}kwd2}{spatial.coverage.by.year.task2}
%
\begin{Usage}
\begin{verbatim}
spatial.coverage.by.year.task2(tdata = t2ce, start.year = 1950, end.year = 2010, which.region = "AT", which.gear = "LL", which.flag = "EU.Portugal")
\end{verbatim}
\end{Usage}
%
\begin{Arguments}
\begin{ldescription}
\item[\code{tdata}] 


\item[\code{start.year}] 


\item[\code{end.year}] 


\item[\code{which.region}] 


\item[\code{which.gear}] 


\item[\code{which.flag}] 


\end{ldescription}
\end{Arguments}
%
\begin{Examples}
\begin{ExampleCode}
##---- Should be DIRECTLY executable !! ----
##-- ==>  Define data, use random,
##--	or do  help(data=index)  for the standard data sets.

## The function is currently defined as
function (tdata = t2ce, start.year = 1950, end.year = 2010, which.region = "AT", 
    which.gear = "LL", which.flag = "EU.Portugal") 
{
    fdata <- tdata[tdata$flagname == which.flag & tdata$geargrpcode == 
        which.gear & tdata$region == which.region, ]
    cl <- 1.5
    ys <- start.year:end.year
    ly <- length(ys)
    for (i in min(ys, na.rm = T):max(ys, na.rm = T)) {
        dat <- fdata[fdata$year == i, ]
        if (length(dat[, 1]) == 0) {
            plot(fdata$longitude, fdata$latitude, type = "n", 
                xaxt = "n", yaxt = "n", ylim = range(tdata$latitude, 
                  na.rm = T), xlim = range(tdata$longitude, na.rm = T))
            map("world", col = "green", fill = T, add = T)
            title(i, cex.main = cl)
        }
        else {
            plot(dat$longitude, dat$latitude, type = "n", xaxt = "n", 
                yaxt = "n", ylim = range(tdata$latitude, na.rm = T), 
                xlim = range(tdata$longitude, na.rm = T))
            points(dat$longitude, dat$latitude, pch = ".", col = "red")
            map("world", add = T, col = "green", fill = T)
            title(i, cex.main = cl)
        }
        mtext(side = 3, outer = T, paste(which.flag, which.gear, 
            sep = " - "))
    }
  }
\end{ExampleCode}
\end{Examples}
\inputencoding{utf8}
\HeaderA{three.d.catch.by.year}{Function to plot spatial distribution of EFFDIS catch data}{three.d.catch.by.year}
\keyword{\textbackslash{}textasciitilde{}kwd1}{three.d.catch.by.year}
\keyword{\textbackslash{}textasciitilde{}kwd2}{three.d.catch.by.year}
%
\begin{Usage}
\begin{verbatim}
three.d.catch.by.year(tdata = task2.lf, what.gear = "LL", what.year = 2005, gridx = 5, gridy = 5, what.species = "alb", what.flag = "All", catchunit = "kg", scaling.f = 1e+06)
\end{verbatim}
\end{Usage}
%
\begin{Arguments}
\begin{ldescription}
\item[\code{tdata}] 


\item[\code{what.gear}] 


\item[\code{what.year}] 


\item[\code{gridx}] 


\item[\code{gridy}] 


\item[\code{what.species}] 


\item[\code{what.flag}] 


\item[\code{catchunit}] 


\item[\code{scaling.f}] 


\end{ldescription}
\end{Arguments}
%
\begin{Examples}
\begin{ExampleCode}
##---- Should be DIRECTLY executable !! ----
##-- ==>  Define data, use random,
##--	or do  help(data=index)  for the standard data sets.

## The function is currently defined as
function (tdata = task2.lf, what.gear = "LL", what.year = 2005, 
    gridx = 5, gridy = 5, what.species = "alb", what.flag = "All", 
    catchunit = "kg", scaling.f = 1e+06) 
{
    if (what.flag == "All") {
        tdata1 <- tdata[tdata$geargrpcode == what.gear & tdata$month < 
            13 & tdata$year == what.year & tdata$species == what.species & 
            tdata$catchunit == catchunit, ]
    }
    else {
        tdata1 <- tdata[tdata$geargrpcode == what.gear & tdata$month < 
            13 & tdata$year == what.year & tdata$species == what.species & 
            tdata$flagname == what.flag & tdata$catchunit == 
            catchunit, ]
    }
    dd <- dim(tdata1)
    ulocs <- length(unique(tdata1$longitude))
    if (dd[1] < 1 | ulocs < 6) {
        print("No Data")
    }
    else {
        coords <- SpatialPointsDataFrame(cbind(x = an(ac(tdata1$longitude)), 
            y = an(ac(tdata1$latitude))), data = tdata1[, c(4, 
            5)])
        geogWGS84 <- CRS("+proj=longlat +ellps=WGS84 +datum=WGS84 +no_defs")
        coords@proj4string <- geogWGS84
        resx <- gridx
        resy <- gridy
        cl <- 1.1
        ca <- 1
        fonts <- 2
        xl <- list(label = "Longitude", font = fonts, cex = cl)
        yl <- list(label = "Latitude", font = fonts, cex = cl)
        zl <- list(font = fonts, cex = cl)
        colintens <- brewer.pal(6, "YlOrRd")
        colland <- brewer.pal(9, "PiYG")[8]
        colgrey <- brewer.pal(9, "Greys")
        figtype <- "tiff"
        parmar <- rep(2, 4)
        paroma <- (c(6, 6, 2, 2) + 0.1)
        reso <- 1
        bbox <- bbox(coords)
        spatBound <- list(xrange = c(floor(range(bbox["x", ])[1]), 
            ceiling(range(bbox["x", ])[2])), yrange = c(floor(range(bbox["y", 
            ])[1]), ceiling(range(bbox["y", ])[2])))
        grd <- createGrid(spatBound$x, spatBound$y, resx, resy, 
            type = "SpatialGridDataFrame", exactBorder = T)
        grd@proj4string <- geogWGS84
        grd@data[] <- 0
        idx <- over(as(coords, "SpatialPoints"), as(grd, "SpatialGrid"))
        tdata1$gridID <- idx
        grd@data[names(table(idx)), 1] <- aggregate(tdata1$measured_catch, 
            by = list(tdata1$gridID), FUN = sum, na.rm = T)$x
        rr <- range(grd@data[an(names(table(idx))), 1])
        cutbreaksval <- list(ALL = c(-1, 0, 10, 25, 50, 100, 
            150, 200))
        legval <- list(ALL = c("0", "1 <= 10", "10 <= 25", "25 <= 50", 
            "50 <= 100", "100 <= 200", "200 <= 400"))
        valdiv <- scaling.f
        unitval <- paste("x", valdiv, catchunit)
        plot(1, 1, col = "white", xlim = spatBound$xrange, ylim = spatBound$yrange, 
            xlab = "", ylab = "", xaxt = "n", yaxt = "n", las = 1, 
            cex.lab = xl$cex, font = xl$font, asp = 1/lonLatRatio(mean(spatBound$xrange), 
                mean(spatBound$yrange)))
        coordGrd <- coordinates(grd)[an(names(table(idx))), ]
        grdPols <- lonLat2SpatialPolygons(lst = lapply(as.list(1:nrow(coordGrd)), 
            function(x) {
                data.frame(SI_LONG = c(coordGrd[x, "s1"] - resx/2, 
                  rep(coordGrd[x, "s1"] + resx/2, 2), coordGrd[x, 
                    "s1"] - resx/2), SI_LATI = c(rep(coordGrd[x, 
                  "s2"] - resy/2, 2), rep(coordGrd[x, "s2"] + 
                  resy/2, 2)))
            }))
        cols <- c("white", colintens)[cut(grd@data[an(names(table(idx))), 
            1]/valdiv, breaks = cutbreaksval$ALL)]
        plot(grdPols, col = cols, add = T, border = "transparent")
        map("world", resolution = 1, add = T, fill = TRUE, col = colland)
        map.axes()
        legend(x = "topright", fill = c("white", colintens), 
            legend = legval$ALL, bg = "white", title = unitval, 
            box.lty = 1)
        title(main = paste(what.flag, what.year, what.gear, what.species, 
            catchunit), outer = F, cex = cl)
        grdPolsDF <- as(grdPols, "SpatialPolygonsDataFrame")
        grdPolsDF@data <- data.frame(value = grd@data[an(names(table(idx))), 
            1], color = cols)
        proj4string(grdPolsDF) <- CRS("+proj=longlat +ellps=WGS84")
    }
  }
\end{ExampleCode}
\end{Examples}
\inputencoding{utf8}
\HeaderA{three.d.effort.by.year}{Function to plot spatial distribution of EFFDIS catch data}{three.d.effort.by.year}
\keyword{\textbackslash{}textasciitilde{}kwd1}{three.d.effort.by.year}
\keyword{\textbackslash{}textasciitilde{}kwd2}{three.d.effort.by.year}
%
\begin{Usage}
\begin{verbatim}
three.d.effort.by.year(tdata = task2.lf, what.gear = "LL", what.year = 2005, gridx = 5, gridy = 5, effort.type = "NO.HOOKS", what.flag = "All", scaling.f = 1e+06)
\end{verbatim}
\end{Usage}
%
\begin{Arguments}
\begin{ldescription}
\item[\code{tdata}] 


\item[\code{what.gear}] 


\item[\code{what.year}] 


\item[\code{gridx}] 


\item[\code{gridy}] 


\item[\code{effort.type}] 


\item[\code{what.flag}] 


\item[\code{scaling.f}] 


\end{ldescription}
\end{Arguments}
%
\begin{Examples}
\begin{ExampleCode}
##---- Should be DIRECTLY executable !! ----
##-- ==>  Define data, use random,
##--	or do  help(data=index)  for the standard data sets.

## The function is currently defined as
function (tdata = task2.lf, what.gear = "LL", what.year = 2005, 
    gridx = 5, gridy = 5, effort.type = "NO.HOOKS", what.flag = "All", 
    scaling.f = 1e+06) 
{
    n0 <- tdata[tdata$dsettype == "n-", ]
    nw <- tdata[tdata$dsettype == "nw", ]
    mm <- duplicated(nw[, -9])
    nw <- nw[mm == TRUE, ]
    w0 <- tdata[tdata$dsettype == "-w", ]
    tdata1 <- rbind(n0, nw, w0)
    tdata1$flagname <- as.character(tdata1$flagname)
    if (what.flag == "All") {
        tdata2 <- tdata1[tdata1$month < 13 & tdata1$year == what.year & 
            tdata1$eff1type == effort.type & tdata1$geargrpcode == 
            what.gear, ]
    }
    if (what.flag != "All") {
        tdata2 <- tdata1[tdata1$month < 13 & tdata1$year == what.year & 
            tdata1$eff1type == effort.type & tdata1$flagname == 
            what.flag & tdata1$geargrpcode == what.gear, ]
    }
    dd <- dim(tdata2)
    ulocs <- length(unique(tdata2$longitude))
    if (dd[1] < 1 | ulocs < 6) {
    }
    else {
        coords <- SpatialPointsDataFrame(cbind(x = an(ac(tdata2$longitude)), 
            y = an(ac(tdata2$latitude))), data = tdata2[, c(4, 
            5)])
        geogWGS84 <- CRS("+proj=longlat +ellps=WGS84 +datum=WGS84 +no_defs")
        coords@proj4string <- geogWGS84
        resx <- gridx
        resy <- gridy
        cl <- 0.8
        ca <- 0.8
        fonts <- 2
        xl <- list(label = "Longitude", font = fonts, cex = cl)
        yl <- list(label = "Latitude", font = fonts, cex = cl)
        zl <- list(font = fonts, cex = cl)
        colintens <- brewer.pal(6, "YlOrRd")
        colland <- brewer.pal(9, "PiYG")[8]
        colgrey <- brewer.pal(9, "Greys")
        figtype <- "tiff"
        parmar <- rep(2, 4)
        paroma <- (c(6, 6, 2, 2) + 0.1)
        reso <- 1
        bbox <- bbox(coords)
        spatBound <- list(xrange = c(floor(range(bbox["x", ])[1]), 
            ceiling(range(bbox["x", ])[2])), yrange = c(floor(range(bbox["y", 
            ])[1]), ceiling(range(bbox["y", ])[2])))
        grd <- createGrid(spatBound$x, spatBound$y, resx, resy, 
            type = "SpatialGridDataFrame", exactBorder = T)
        grd@proj4string <- geogWGS84
        grd@data[] <- 0
        idx <- over(as(coords, "SpatialPoints"), as(grd, "SpatialGrid"))
        tdata2$gridID <- idx
        grd@data[names(table(idx)), 1] <- aggregate(tdata2$eff1, 
            by = list(tdata2$gridID), FUN = sum, na.rm = T)$x
        rr <- range(grd@data[an(names(table(idx))), 1])
        cutbreaksval <- list(ALL = c(-1, 0, 10, 25, 50, 100, 
            150, 200))
        legval <- list(ALL = c("0", "1 <= 10", "10 <= 25", "25 <= 50", 
            "50 <= 100", "100 <= 200", "200 <= 400"))
        valdiv <- scaling.f
        unitval <- paste("x", valdiv, "effort units")
        plot(1, 1, col = "white", xlim = spatBound$xrange, ylim = spatBound$yrange, 
            xlab = "", ylab = "", xaxt = "n", yaxt = "n", las = 1, 
            cex.lab = xl$cex, font = xl$font, asp = 1/lonLatRatio(mean(spatBound$xrange), 
                mean(spatBound$yrange)))
        coordGrd <- coordinates(grd)[an(names(table(idx))), ]
        grdPols <- lonLat2SpatialPolygons(lst = lapply(as.list(1:nrow(coordGrd)), 
            function(x) {
                data.frame(SI_LONG = c(coordGrd[x, "s1"] - resx/2, 
                  rep(coordGrd[x, "s1"] + resx/2, 2), coordGrd[x, 
                    "s1"] - resx/2), SI_LATI = c(rep(coordGrd[x, 
                  "s2"] - resy/2, 2), rep(coordGrd[x, "s2"] + 
                  resy/2, 2)))
            }))
        cols <- c("white", colintens)[cut(grd@data[an(names(table(idx))), 
            1]/valdiv, breaks = cutbreaksval$ALL)]
        plot(grdPols, col = cols, add = T, border = "transparent")
        map("world", resolution = 1, add = T, fill = TRUE, col = colland)
        map.axes()
        legend(x = "topright", fill = c("white", colintens), 
            legend = legval$ALL, bg = "white", title = unitval, 
            box.lty = 1)
        title(main = paste(what.flag, what.year, what.gear, effort.type), 
            outer = F, cex = cl)
        grdPolsDF <- as(grdPols, "SpatialPolygonsDataFrame")
        grdPolsDF@data <- data.frame(value = grd@data[an(names(table(idx))), 
            1], color = cols)
        proj4string(grdPolsDF) <- CRS("+proj=longlat +ellps=WGS84")
    }
  }
\end{ExampleCode}
\end{Examples}
\inputencoding{utf8}
\HeaderA{trend}{Function to calculate absolute time from month and year}{trend}
\keyword{\textbackslash{}textasciitilde{}kwd1}{trend}
\keyword{\textbackslash{}textasciitilde{}kwd2}{trend}
%
\begin{Usage}
\begin{verbatim}
trend(year, month, start.year = 1958)
\end{verbatim}
\end{Usage}
%
\begin{Arguments}
\begin{ldescription}
\item[\code{year}] 


\item[\code{month}] 


\item[\code{start.year}] 


\end{ldescription}
\end{Arguments}
%
\begin{Examples}
\begin{ExampleCode}
##---- Should be DIRECTLY executable !! ----
##-- ==>  Define data, use random,
##--	or do  help(data=index)  for the standard data sets.

## The function is currently defined as
function (year, month, start.year = 1958) 
{
    nyear = year - start.year
    trend = month + nyear * 12
    trend
  }
\end{ExampleCode}
\end{Examples}
\inputencoding{utf8}
\HeaderA{yr.month.coverage.task2}{Function to plot data distribution as a function of year and month}{yr.month.coverage.task2}
\keyword{\textbackslash{}textasciitilde{}kwd1}{yr.month.coverage.task2}
\keyword{\textbackslash{}textasciitilde{}kwd2}{yr.month.coverage.task2}
%
\begin{Usage}
\begin{verbatim}
yr.month.coverage.task2(tdata = t2ce, start.year = 1950, end.year = 2010, which.gear = "LL", which.region = "AT", which.flag = "EU.Portugal")
\end{verbatim}
\end{Usage}
%
\begin{Arguments}
\begin{ldescription}
\item[\code{tdata}] 


\item[\code{start.year}] 


\item[\code{end.year}] 


\item[\code{which.gear}] 


\item[\code{which.region}] 


\item[\code{which.flag}] 


\end{ldescription}
\end{Arguments}
%
\begin{Examples}
\begin{ExampleCode}
##---- Should be DIRECTLY executable !! ----
##-- ==>  Define data, use random,
##--	or do  help(data=index)  for the standard data sets.

## The function is currently defined as
function (tdata = t2ce, start.year = 1950, end.year = 2010, which.gear = "LL", 
    which.region = "AT", which.flag = "EU.Portugal") 
{
    n0 <- tdata[tdata$dsettype == "n-", ]
    nw <- tdata[tdata$dsettype == "nw", ]
    mm <- duplicated(nw[, -9])
    nw <- nw[mm == TRUE, ]
    w0 <- tdata[tdata$dsettype == "-w", ]
    tdata1 <- rbind(n0, nw, w0)
    tdata1 <- tdata[tdata$month < 13, ]
    fdata <- tdata1[tdata1$flagname == which.flag & tdata1$geargrpcode == 
        which.gear & tdata1$region == which.region, ]
    dd <- dim(fdata)
    if (dd[1] == 0) {
    }
    else {
        fmat <- matrix(NA, length(1950:2015), 12)
        dimnames(fmat) <- list(c(1950:2015), 1:12)
        ymc <- table(fdata$year, fdata$month)
        dimnames(ymc)[[1]] <- sort(unique(fdata$year))
        dimnames(ymc)[[2]] <- month.abb
        mm <- match(dimnames(ymc)[[1]], dimnames(fmat)[[1]])
        fmat[mm, ] <- ymc
        image(1950:2015, 1:12, fmat, xaxt = "n", yaxt = "n", 
            xlab = "", ylab = "", col = terrain.colors(100), 
            xlim = c(start.year, end.year), ylim = range(fdata$month, 
                na.rm = T))
        contour(1950:2015, 1:12, fmat, add = T)
        axis(side = 1, at = start.year:end.year, label = as.character(start.year:end.year))
        ms <- range(fdata$month, na.rm = T)
        axis(side = 2, at = ms[1]:ms[2], label = month.abb[ms[1]:ms[2]])
        title(paste(which.flag, which.gear, sep = " - "))
    }
  }
\end{ExampleCode}
\end{Examples}
\printindex{}
\end{document}
